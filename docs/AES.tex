\documentclass[]{book}
\usepackage{lmodern}
\usepackage{amssymb,amsmath}
\usepackage{ifxetex,ifluatex}
\usepackage{fixltx2e} % provides \textsubscript
\ifnum 0\ifxetex 1\fi\ifluatex 1\fi=0 % if pdftex
  \usepackage[T1]{fontenc}
  \usepackage[utf8]{inputenc}
\else % if luatex or xelatex
  \ifxetex
    \usepackage{mathspec}
  \else
    \usepackage{fontspec}
  \fi
  \defaultfontfeatures{Ligatures=TeX,Scale=MatchLowercase}
\fi
% use upquote if available, for straight quotes in verbatim environments
\IfFileExists{upquote.sty}{\usepackage{upquote}}{}
% use microtype if available
\IfFileExists{microtype.sty}{%
\usepackage{microtype}
\UseMicrotypeSet[protrusion]{basicmath} % disable protrusion for tt fonts
}{}
\usepackage[margin=1in]{geometry}
\usepackage{hyperref}
\hypersetup{unicode=true,
            pdftitle={Animal Environmental Science},
            pdfauthor={Sangrak Lee; Youngjun Na},
            pdfborder={0 0 0},
            breaklinks=true}
\urlstyle{same}  % don't use monospace font for urls
\usepackage{natbib}
\bibliographystyle{apalike}
\usepackage{longtable,booktabs}
\usepackage{graphicx,grffile}
\makeatletter
\def\maxwidth{\ifdim\Gin@nat@width>\linewidth\linewidth\else\Gin@nat@width\fi}
\def\maxheight{\ifdim\Gin@nat@height>\textheight\textheight\else\Gin@nat@height\fi}
\makeatother
% Scale images if necessary, so that they will not overflow the page
% margins by default, and it is still possible to overwrite the defaults
% using explicit options in \includegraphics[width, height, ...]{}
\setkeys{Gin}{width=\maxwidth,height=\maxheight,keepaspectratio}
\IfFileExists{parskip.sty}{%
\usepackage{parskip}
}{% else
\setlength{\parindent}{0pt}
\setlength{\parskip}{6pt plus 2pt minus 1pt}
}
\setlength{\emergencystretch}{3em}  % prevent overfull lines
\providecommand{\tightlist}{%
  \setlength{\itemsep}{0pt}\setlength{\parskip}{0pt}}
\setcounter{secnumdepth}{5}
% Redefines (sub)paragraphs to behave more like sections
\ifx\paragraph\undefined\else
\let\oldparagraph\paragraph
\renewcommand{\paragraph}[1]{\oldparagraph{#1}\mbox{}}
\fi
\ifx\subparagraph\undefined\else
\let\oldsubparagraph\subparagraph
\renewcommand{\subparagraph}[1]{\oldsubparagraph{#1}\mbox{}}
\fi

%%% Use protect on footnotes to avoid problems with footnotes in titles
\let\rmarkdownfootnote\footnote%
\def\footnote{\protect\rmarkdownfootnote}

%%% Change title format to be more compact
\usepackage{titling}

% Create subtitle command for use in maketitle
\newcommand{\subtitle}[1]{
  \posttitle{
    \begin{center}\large#1\end{center}
    }
}

\setlength{\droptitle}{-2em}

  \title{Animal Environmental Science}
    \pretitle{\vspace{\droptitle}\centering\huge}
  \posttitle{\par}
    \author{Sangrak Lee; Youngjun Na}
    \preauthor{\centering\large\emph}
  \postauthor{\par}
      \predate{\centering\large\emph}
  \postdate{\par}
    \date{Last update: 2019-03-11}

\usepackage{booktabs}

\begin{document}
\maketitle

{
\setcounter{tocdepth}{1}
\tableofcontents
}
\chapter*{Welcome}\label{welcome}
\addcontentsline{toc}{chapter}{Welcome}

This is the website for \textbf{``Animal environmental science''}. To
understanding individual animals, we have to understand the relationship
they have with their environment. This book will focus at the
interaction between animals and the environment.

This website is (and will always be) \textbf{free to use}, and is
licensed under the
\href{http://creativecommons.org/licenses/by-nc-nd/3.0/us/}{Creative
Commons Attribution-NonCommercial-NoDerivs 3.0} License. The book is
written in \href{https://rmarkdown.rstudio.com}{RMarkdown} with
\href{https://bookdown.org}{bookdown}. Some photographs used in this
book was from \href{https://unsplash.com/}{Unsplash.com}. If you click
the download button up above, you can download the PDF version of the
book.

\includegraphics{figures/monkey.jpeg}\\
(Snow Monkey Niseko, Kutchan-chō, Japan)

\chapter{Introduction}\label{intro}

All living creatures constantly interact with the environment. To
understanding individual animals, we have to understand the relationship
they have with their environment. Also, animals affect the environment.
From birth to death, an animal generates carbon dioxide, methane, feces,
and urine. The excretes from the animal are built with molecules such as
carbon, nitrogen, sulfur, and phosphorus, and are recycled within and
between ecosystems.

\begin{figure}

{\centering \includegraphics[width=1\linewidth]{figures/polar} 

}

\caption{Alaskan Malamute has the heat-conserving features. To retain heat you want the least body surface area compared to body volume.}\label{fig:snow-dog}
\end{figure}

\chapter{Animal and environment}\label{chapter2}

\section{External environment}\label{external-environment}

Animal never separates from the stimuli from outside. Basically, animals
can find food, shelter, protection, and mates from the environment
called \emph{habitat}. The animal habitat includes both phisical
(non-living) and biotic (livinig) components (Table \ref{tab:habitat}).

\begin{table}[t]

\caption{\label{tab:habitat}Components of habitat (physical and biotic)}
\centering
\begin{tabular}{ll}
\toprule
Physical & Biotic\\
\midrule
Temperature & Plant matter\\
Humidity & Predators\\
Oxygen & Parasites\\
Wind & Competitors\\
Soil & Individuals of the same species\\
\addlinespace
Light intensity & \\
Elevation & \\
\bottomrule
\end{tabular}
\end{table}

Animal habitat is constantly changed over time. Not only natural
disasters (eruption of volcano, earthquake, tsunami, and wildfire), also
human activity can affect the animal habitat. Unlike the wildlife, the
environment of domesticated animals (such as cow, pig, poultry, and dog)
that raised in the facility are controlled by the human. In the domestic
animals, the external environment includes both physical (e.g.~housing,
feeder, paddock, fence, and noise) and biotic (e.g.~human, mate, and
feed ingredients) components.

\subsection{Biome}\label{biome}

A biome is a community of plants and animals that have common
characteristics for the environment they exist in (Figure
\ref{fig:biomes}). They can be found over a range of continents. Biomes
are distinct biological communities that have formed in response to a
shared physical climate.

The principal biome-types are:

\begin{enumerate}
\def\labelenumi{\arabic{enumi}.}
\tightlist
\item
  Tundra
\item
  Taiga
\item
  Deciduous forest
\item
  Grasslands
\item
  Desert
\item
  High plateaus
\item
  Tropical forest
\item
  Minor terrestrial biomes
\end{enumerate}

\begin{figure}

{\centering \includegraphics[width=1\linewidth]{figures/biomes} 

}

\caption{Mapping terrestrial biomes around the world}\label{fig:biomes}
\end{figure}

Holdridge (1947; 1967) classified climates based on the biological
effects of temperature and rainfall on vegetation under the assumption
that these two abiotic factors are the largest determinants of the types
of vegetation found in a habitat (Figure \ref{fig:holdridge}).

The three axes of the barycentric subdivisions are:

\begin{enumerate}
\def\labelenumi{\arabic{enumi}.}
\tightlist
\item
  Precipitation (annual, logarithmic)
\item
  Biotemperature (mean annual, logarithmic)
\item
  Potential evapotranspiration ratio (PET) to mean total annual
  precipitation.
\end{enumerate}

Further indicators incorporated into the system are:

\begin{enumerate}
\def\labelenumi{\arabic{enumi}.}
\tightlist
\item
  Humidity provinces
\item
  Latitudinal regions
\item
  Altitudinal belts
\end{enumerate}

\begin{figure}

{\centering \includegraphics[width=1\linewidth]{figures/lifezones} 

}

\caption{Holdridge life zone classification scheme.}\label{fig:holdridge}
\end{figure}

\section{Internal environment}\label{internal-environment}

\begin{quote}
``The living body, though it has need of the surrounding environment, is
nevertheless relatively independent of it.'' --- Claude Bernard
\end{quote}

Higher animals have complex systems that respond to stimuli to perform
their essential body functions. When the animal receives the signals
from the sensory organs, they produce a local reflex action and/or react
in the central nervous system. Weak signals produce no responses, but
strong stimuli change the physiological or behavioral status of the
animal.

\begin{figure}

{\centering \includegraphics[width=0.6\linewidth]{figures/animal-env} 

}

\caption{External and internal environment}\label{fig:ext-int-env}
\end{figure}

\subsection{Shelford's law of
tolerance}\label{shelfords-law-of-tolerance}

\begin{quote}
``Each and every species is able to exist and reproduce successfully
only within a definite range of environmental conditions.'' --- Ronald
Good
\end{quote}

Although external environments are continuously changed, if animals in
the normal status, they keep the composition of the extracellular fluid
(internal environment) constant to maintain their life. We call it
\emph{homeostasis}.

\begin{table}[t]

\caption{\label{tab:homeostasis}List of homeostatic control variables}
\centering
\begin{tabular}{l}
\toprule
Control variables\\
\midrule
Core temperature; Blood glucose; Iron levels; Copper regulation; Levels of blood gases;\\
Blood oxygen content; Arterial blood pressure; Calcium levels; Sodium concentration;\\
Potassium concentration; Fluid balance; Blood pH; Cerebrospinal fluid; Neurotransmission;\\
Neuroendocrine system; Gene regulation; and Energy balance\\
\bottomrule
\end{tabular}
\end{table}

However, the capacity to maintain the homeostasis is broken when the
animals let the harsh environments and differ by their species.
\textbf{Animals may be limited in their growth and their occurrence by
the minimum, maximum, and optimum condition} \citep{shelford} (Fig.
\ref{fig:law-of-tol}).

\begin{figure}

{\centering \includegraphics{AES_files/figure-latex/law-of-tol-1} 

}

\caption{Shelford's law of tolerance}\label{fig:law-of-tol}
\end{figure}

The optimum range of environmental condition may differ within the same
organism, and it is not necessarily fixed. They can change as:

\begin{itemize}
\tightlist
\item
  Change of seasons
\item
  Change of environmental conditions
\item
  Life stage of the organism
\end{itemize}

\subsection{Adaptation}\label{adaptation}

\begin{quote}
``Changes in morphological, anatomical, physiological, biochemical and
behavioral characteristics of the animal which promote welfare and favor
survival in a specific environment.'' --- Hafez
\end{quote}

\citet{hafez1968adaptation} defined an adaptation as above. The
adaptation helps an animal survive in their external environment. The
representative adaptive traits are:

\begin{enumerate}
\def\labelenumi{\arabic{enumi}.}
\tightlist
\item
  Structural adaptation
\item
  Behavioral adaptation
\item
  Physiological adaptation
\end{enumerate}

Structural adaptation is the changes in physical features (e.g.~body
shape, skin, and internal organs) of the animal. Behavioral adaptation
is the changes in behaviors (e.g.~searching for food, mating,
vocalizations, and mitigation) of the animal. Physiological adaptation
is the changes in the animal body functions such as growth, temperature
regulation, and ionic balance. Sometimes, adapted animal create a new
species (\emph{speciation}).

\begin{figure}

{\centering \includegraphics[width=1\linewidth]{figures/adaptation} 

}

\caption{Migration is an example of a behavioral adaptation. Grey whales swim from the Arctic to Mexico every year.}\label{fig:adaptation}
\end{figure}

\subsection{Acclimatization}\label{acclimatization}

Acclimatization is the physiological changes induced by a complex of
factors such as altitude, temperature, humidity, photoperiod, or pH.
Acclimatization is the short-term process (hours to weeks) by comparison
with adaptation (take place over many generations).

\chapter{Temperature}\label{temperature}

Temperature is a quantity expressing of the amount of heat. Because a
rate of every chemical reaction occurs in the animal's body is affected
by the temperature, it is a very important factor to all animals. Like
most chemical reactions, an enzyme-catalyzed reaction rate in the
animal's body increases as the temperature is raised. However, extremely
high or low temperature results in loss of activity or lose the
structure for most enzymes (\emph{denaturation}; Figure \ref{fig:q10}).

\begin{figure}

{\centering \includegraphics[width=0.6\linewidth]{figures/q10} 

}

\caption{The effects of temperature on enzyme activity [@q10]. Top - increasing temperature increases the rate of reaction (Q10 coefficient). Middle - the fraction of folded and functional enzyme decreases above its denaturation temperature. Bottom - consequently, an enzyme's optimal rate of reaction is at an intermediate temperature.}\label{fig:q10}
\end{figure}

\section{Poikilotherm and homeotherm}\label{poikilotherm-and-homeotherm}

Key factors for animal surviving are to adapt to external environmental
changes and maintain a consistent internal environment. The animal can
be divided into two types for response to external temperatures:
\emph{poikilotherm} (cold-blooded animals) and \emph{homeotherm}
(warm-blooded animals). Examples of poikilotherms are most fish,
amphibians, and reptiles. Their internal body temperature varies
considerably according to their external environments. On the other
hand, homeotherm maintains their thermal homeostasis regardless of the
external temperature. The examples of homeotherm are birds and mammals.

\begin{figure}

{\centering \includegraphics{AES_files/figure-latex/body-temp-comparision-1} 

}

\caption{Comparison of body temperature response by snake (poikiloterm) and bobcat (homeoterm) to changing ambient temperature.}\label{fig:body-temp-comparision}
\end{figure}

\subsection{Poikilotherm}\label{poikilotherm}

The term derives from the acient Greek language \emph{poikilos}
(ποικίλος; changeable) and \emph{thermos} (θερμός; heat). The body
temperature of poikilotherms varies considerably than those of
homeotherms (Figure \ref{fig:body-temp-comparision}). They generally use
solar radiation for maintaining their body temperature and have four to
ten enzyme systems that can operate at different ambient temperature
because the temperature affects the chemical reactions.

\begin{figure}

{\centering \includegraphics[width=1\linewidth]{figures/flog} 

}

\caption{Green frog on blue surface.}\label{fig:flor}
\end{figure}

\subsection{Homeoterm}\label{homeoterm}

Homeotherms can maintain body temperature independently from ambient
temperatures by regulating the metabolic process. They preserve their
body temperature by muscle contraction and brown adipose tissue is
catabolized for heat production \citep{grigg2004evolution}. In hot
environments, they use evaporative cooling (sweating or panting) for
maintaining their body temperature. Most of the domestic animals are
homeotherm.

\begin{figure}

{\centering \includegraphics[width=1\linewidth]{figures/energy-system} 

}

\caption{Overview of feed energy flow through the animal body}\label{fig:e-system}
\end{figure}

\begin{figure}

{\centering \includegraphics[width=1\linewidth]{figures/kimetal} 

}

\caption{Infrared cameras image that cows generating the heats. @kim2018image developed the algorithms for tracking the cows using IR camera video.}\label{fig:kimetal}
\end{figure}

In some homeoterms (bears, hedgehog, marmot, and so on) and
poikilotherms (frogs, turtles, snake, and so on), they can enter the
\emph{hibernation} in the cold season: the body temperature is dropped,
and the metabolic rate is depressed. Hibernating bears can recycle their
body proteins and urine to avoid muscle loss.

\subsection{Heterotherm}\label{heterotherm}

Heterotherms exhibit the characteristics of both poikilotherm and
homeotherm. They can switch between poikilothermic and homeothermic
strategies. In some bat species, for example, body temperature and
metabolic rate are elevated only when they are active. When they at
rest, metabolic rate is drastically dropped thereby the body temperature
is decreased to the ambient temperature.

\section{Thermoregulation}\label{thermoregulation}

Thermoregulation is a process to maintain the internal temperature
within certain boundaries. In homeotherms, thermoregulatory physiology
is mainly controlled by nervous and endocrine systems. The core
temperature of the animal is primarily regulated by the hypothalamus. If
the ambient temperature is going to cold, they generate heat via
metabolic processes to keep their body temperature. In contrast, in hot
conditions, sweat glands release sweat for evaporates and the blood
vessels going to wider for increasing the blood flow to the skin.

\begin{table}[t]

\caption{\label{tab:norm-body-temp}Normal body temperature of the domestic animals; Body temperatures may be 1°C above or below these temperatures.}
\centering
\begin{tabular}{llll}
\toprule
Animal & Normal temperature (°C) & Animal & Normal temerature (°C)\\
\midrule
Cattle & 38.5 & Donkey & 38.2\\
Calf & 39.5 & Chicken & 42.0\\
Buffalo & 38.2 & Camel & 34.5-41.0\\
Sheep & 39.0 & Horse & 38.0\\
Llama, alpaca & 38.0 & Pig & 39.0\\
\addlinespace
Goat & 39.5 & Piglet & 39.8\\
\bottomrule
\end{tabular}
\end{table}

In poikilotherms, they use external sources of temperature to keep their
body temperatures (Table \ref{tab:cooling}). To regulate their body
temperature, they sometimes climbing the trees, entering the warm water,
lying on the cool ground, or lying in the sun. There are some methods
for thermoregulation in poikilotherms: \emph{convection},
\emph{conduction}, and \emph{radiation}. Convection is the transfer of
heat via the movement of molecules within fluids (gases or liquids).
Conduction is the transfer of heat via the direct molecular collision.
Radiation is the transfer of heat in the form of waves or particles
(sunlight is the most familiar forms of radiation). Once there's a
thermal equilibrium between the animal and environment, the thermal
exchange will be stopped.

\begin{table}[t]

\caption{\label{tab:cooling}Cooling and heating methods for poikilotherms}
\centering
\begin{tabular}{lll}
\toprule
Methods & Cooling & Heating\\
\midrule
Convection & Increasing blood flow to body surfaces & Entering a warm water or air current; Building an insulated nest or burrow\\
Conduction & Lying on cool ground; Staying wet in a river, lake or sea; Covering in cool mud & Lying on a hot surface\\
Radiation & Get away from the sun & Lying in the sun; Folding skin to reduce exposure\\
\bottomrule
\end{tabular}
\end{table}

\section{Temperature humidity index
(THI)}\label{temperature-humidity-index-thi}

The productivity of domestic animals is primarily affected by air
temperature, and altered by wind, humidity, and radiation.
Temperature--humidity index (THI) is a combination of temperature and
humidity that is a measure of the degree of discomfort experienced by an
individual in warm weather (a.k.a. discomfort index). This unitless
index was first introduced by Thom (1959) to describe the effect of
ambient temperature on humans but has been adapted to describe thermal
conditions that drive heat stress in dairy cattle (De Rensis et al.,
2015). Temperature-humidity index for dairy cow is calculated as

\(THI = (0.8*T) + [H*(T - 14.4)] + 46.4\)

where T is the air temperature and H is the relative humidity. The THI
is a useful tool for predicting the heat stress of cows, however, it
does not account for solar radiation and wind speed which can affect the
heat load of cattle.

\begin{figure}

{\centering \includegraphics[width=1\linewidth]{figures/THI} 

}

\caption{THI chart for dairy cows. Yellow = Stress Threshold Respiration rate exceeds 60 BPM. Milk yield losses begin. Repro losses detectable. Rectal Temperature exceeds 38.5°C (101.3°F). Orange = Mild-Moderate Stress Respiration Rate Exceeds 75 BPM. Rectal Temperature exceeds 39°C (102.2°F). Red = Moderate-Severe Stress Respiration Rate Exceeds 85 BPM Rectal Temperature exceeds 40 °C (104°F). Purple = Severe Stress. Respiration Rate 120-140 BPM. Rectal Temperature exceeds 41 °C (106°F)}\label{fig:THI}
\end{figure}

\section{Effects heat stress on the animal production and
health}\label{effects-heat-stress-on-the-animal-production-and-health}

Homeotherms have optimal temperature zones for production within which
no additional energy above maintenance is expended to heat or cool the
body. The range for lactating dairy cows is estimated to be from −0.5 to
20°C (Johnson, 1987).

\subsection{Dairy cattle}\label{dairy-cattle}

\begin{figure}

{\centering \includegraphics[width=1\linewidth]{figures/heatstress-dairy} 

}

\caption{The relationship between the immediate effects of environmental heat stress and the 3 key constructs of animal welfare: (1) the biological functioning (and health) of the animal, (2) the affective states the animal is experiencing, and (3) the naturalness of its life under current heat management strategies (Polsky et al., 2017).}\label{fig:heat-dairy}
\end{figure}

\textbf{Heat stress decreases milk production.} Lactating dairy cows
have an increased sensitivity to heat stress compared with nonlactating
(dry) cows, due to milk production elevating metabolism (Purwanto et
al., 1990). Moreover, because of the positive relationship between milk
yield and heat production, higher yielding cows are more challenged by
heat stress than lower yielding animals (Spiers et al., 2004).

When a cow becomes heat stressed, an immediate coping mechanism is to
reduce DMI, causing a decrease in the availability of nutrients used for
milk synthesis (West, 2003; Rhoads et al., 2009). Simultaneously, there
is an increase in basal metabolism caused by activation of the
thermoregulatory system. Mild to severe heat stress can increase
metabolic maintenance requirements by 7 to 25\% (NRC, 2001).

\textbf{Heat stress decreases reproductivity}. The decrease in
conception rates during summer seasons can range between 20 and 30\%,
with evident seasonal patterns of estrus detection (De Rensis and
Scaramuzzi, 2003). Elevated environmental temperatures negatively affect
the cow's ability to display natural mating behavior, as it reduces both
the duration and intensity of estrous expression (Orihuela, 2000). A
reduction in estrous behavior has been argued to be the result of
reduced DMI and the subsequent effects on hormone production (Westwood
et al., 2002). Decreased milk production and declining reproductive
success are the most commonly examined components of a heat-stressed
dairy cow's health.

Alterations in housing and management strategies have attempted to
mitigate the heat stress. Basically, various cooling options for dairy
cows exist based on the principles of convection, conduction, radiation,
and evaporation.

\begin{enumerate}
\def\labelenumi{\arabic{enumi}.}
\item
  \textbf{Fan installations}, which facilitate air movement and increase
  convection, have been used to reduce environmental temperatures and
  mitigate heat stress by decreasing respiratory rate and rectal
  temperature and increasing DMI (Armstrong, 1994).
\item
  \textbf{High-pressure mist injected into fans} (which function to cool
  the microclimate air that the cows inspire) or large water droplets
  from low-pressure sprinkler systems that completely wet the cow by
  soaking the hair coat.
\item
  \textbf{Physical structures} that provide shade such as trees, roofs,
  or cloth can create more hospitable microclimates for cows because of
  the reduction in solar radiation exposure and decline in ambient
  temperature.
\item
  \textbf{Barn orientation} (depending on geographic location) can also
  help mitigate heat stress by reducing the insolation and stall surface
  temperature.
\end{enumerate}

\subsection{Beef cattle}\label{beef-cattle}

As temperatures heat up during the summer cattle producers need to
assess the heat stress that their cattle are under. Typically pastured
cattle are not as susceptible to heat stress as feedlot cattle. Pastured
cattle have the ability to seek shade, water and air movement to cool
themselves.

Compared to other animals cattle cannot dissipate their heat load very
effectively. Cattle do not sweat effectively and rely on respiration to
cool themselves. Cattle should not be worked during times of extreme
heat and only early in morning when it is hot.

Cattle's core temperature peaks 2 hours after peak environmental
temperature. It also takes at least 6 hours for cattle to dissipate
their heat load. Therefore, if peak temperature occurred at 4:00 pm
cattle will not have recovered from that heat load until after 12:00 am
and it will be later than that before cattle have fully recovered from
the entire days heat load. Special attention should be paid to cattle
with increased risk of heat stress including heavy cattle, black cattle
and respiratory compromised animals.

\subsubsection{Heat stress management}\label{heat-stress-management}

\begin{enumerate}
\def\labelenumi{\arabic{enumi}.}
\item
  The water requirements of cattle increases during heat stress. Cattle
  lose water from increased respiration and perspiration.
  \textbf{Consumption of water} is the quickest method for cattle to
  reduce their core body temperature.
\item
  Heat production from feed intake peaks 4 to 6 hours after feeding.
  Therefore heat production in cattle fed in the morning will peak in
  the middle of the day when environmental temperatures are also
  elevated. \textbf{Changing the ration} indicates that lowering the
  energy content of diet will decrease the heat load. The general
  recommendation is to reduce the diet energy content by 5 to 7\%.
\item
  \textbf{Increasing the air flow} can help cattle cope with extreme
  heat events. Wind speed has been shown to be associated with ability
  of cattle to regulate their heat load.
\item
  \textbf{Sprinklers} can be used to cool cattle during times of stress.
  Sprinklers increase evaporative cooling and can reduce ground
  temperature. Sprinklers should thoroughly wet the animal and not just
  mist the air in order to cool the animal. Sprinklers should be placed
  away from feed bunks and waterers. Cattle need to be introduced to
  sprinklers prior to extreme heat.
\end{enumerate}

\subsection{Swine}\label{swine}

Most animals can transfer internal heat to the outside of the body by
sweating and panting. These are the two most important tools for the
maintenance of body temperature and form their inbuilt evaporative
cooling system. However, pigs do not sweat and have relatively small
lungs. Due to these physiological limitations and their relatively thick
subcutaneous fat, pigs are prone to heat stress. Today's modern pig
genotypes produce considerably more heat than their predecessors (new
genetic lines of pigs produce nearly 20\% more heat than their
counterparts in the early 1980s.).

The two obvious symptoms observed when pigs are exposed to heat stress
are increased respiration rate and loss of appetite. If the pig exposed
to 35°C for 24 hours significantly damaged the intestinal defense system
and also increased plasma endotoxin levels. It can provide an
opportunity for infection as pathogenic bacteria can invade the body
more easily.

\subsubsection{Heat stress management}\label{heat-stress-management-1}

\begin{itemize}
\tightlist
\item
  Increase ventilation and airflow and regularly check cooling system.
\item
  Reduce stocking density if possible.
\item
  Maintain drinking water temperature as low as possible (around 10°C is
  ideal but difficult to achieve).
\item
  Avoid feeding between 10:00 to 16:00 (the hottest period of the day).
\item
  Supplement electrolytes and antioxidants through the water supply.
\item
  Minimise excess non-essential amino acids and fibre (minimising
  intestinal fermentation and therefore heat production).
\item
  Increase availability of antioxidants through the diet such as vitamin
  E and betaine.
\end{itemize}

\subsection{Poultry}\label{poultry}

High temperature affects the physiological functions of poultry birds at
any stage of life which in results affects the poultry production
performance. Modern commercial poultry produces more body heat due to
their fast metabolism. This makes birds more sensitive to environmental
temperature. In addition, the chicks are highly sensitive to heat stress
because they don't have sweat glands.

\subsubsection{Heat stress management}\label{heat-stress-management-2}

\begin{itemize}
\tightlist
\item
  Semi-open buildings can help the ventilation.
\item
  Maintain drinking water temperature as low as possible.
\item
  A shiny surface reflects solar radiation more than a dark or rusty
  roof.
\item
  Fat addition and excess essential amino acids in feed.
\item
  Supplement of minerals (Fe, Zn, Se and Cr) and vitamins (vitamin A, C
  and E).
\item
  Genetic selection strategies.
\end{itemize}

\subsection{Canine}\label{canine}

When a dog is exposed to high temperatures, heat stroke or heat
exhaustion can result. Heat stroke is a very serious condition that
requires immediate medical attention. Dogs do not sweat through their
skin like humans. They release heat primarily by panting and they sweat
through the foot pads and nose. If a dog cannot effectively expel heat,
the internal body temperature begins to rise. Once the dog's temperature
reaches 41°C damage to the body's cellular system and organs may become
irreversible.

Signs of heat stroke are 1) increasing the rectal temperature, 2)
vigorous panting, 3) dark red gums, 4) tacky or dry mucus membranes
(specifically the gums), 5) lying down and unwilling (or unable) to get
up, and/or 6) dizziness or disorientation.

\subsubsection{Treatments for heat
stroke}\label{treatments-for-heat-stroke}

\begin{enumerate}
\def\labelenumi{\arabic{enumi}.}
\tightlist
\item
  Move your dog out of the heat and away from the sun right away.\\
\item
  Begin cooling your dog by placing cool, wet rags or washcloths on the
  body.\\
\item
  Do not use ice or cold water. Extreme cold can cause the blood vessels
  to constrict, preventing the body's core from cooling and actually
  causing the internal temperature to further rise. When the body
  temperature reaches 39.5°C, stop cooling.
\item
  Offer your dog cool water, but do not force water into your dog's
  mouth.
\end{enumerate}

\subsubsection{Preventing the heat
stroke}\label{preventing-the-heat-stroke}

\begin{enumerate}
\def\labelenumi{\arabic{enumi}.}
\tightlist
\item
  Never leave your dog alone in the car on a warm day, regardless of
  whether the windows are open. Even if the weather outside is not
  extremely hot, the inside of the car acts like an oven.\\
\item
  Avoid vigorous exercise on warm days.\\
\item
  Keep fresh cool water available at all times.\\
\item
  Certain types of dogs are more sensitive to heat -- especially obese
  dogs and short-nosed breeds, like Pugs and Bulldogs (Figure
  \ref{fig:heat-canine}).
\end{enumerate}

\begin{figure}

{\centering \includegraphics[width=1\linewidth]{figures/pug} 

}

\caption{Certain types of dogs are more sensitive to heat – especially obese dogs and short-nosed breeds, like Pugs and Bulldogs.}\label{fig:heat-canine}
\end{figure}

\chapter{Light}\label{light}

\section{Photoperiodic response}\label{photoperiodic-response}

\section{Effects on productivity}\label{effects-on-productivity}

\subsection{Wool}\label{wool}

\subsection{Feathers}\label{feathers}

\subsection{Antlers}\label{antlers}

\subsection{Puberty}\label{puberty}

\subsection{Reproduction}\label{reproduction}

\subsection{Behavior}\label{behavior}

\subsection{Light control in poultry
production}\label{light-control-in-poultry-production}

\chapter{Sound}\label{sound}

\chapter{Air quality}\label{air-quality}

\chapter{Water quality}\label{water-quality}

\chapter{Cycles of materials}\label{cycles-of-materials}

\section{Ecosystem}\label{ecosystem}

\section{Trophic level}\label{trophic-level}

\section{Carbon cycle}\label{carbon-cycle}

\section{Nitrogen cycle}\label{nitrogen-cycle}

\section{Calcium and Phosphorus
cycle}\label{calcium-and-phosphorus-cycle}

\chapter{Manure}\label{manure}

\section{Charateristics of animal
manure}\label{charateristics-of-animal-manure}

\section{Manure treatment}\label{manure-treatment}

\subsection{Solid fertilizer
(Composting)}\label{solid-fertilizer-composting}

\subsection{Liquid fertilizer}\label{liquid-fertilizer}

\subsection{Purification}\label{purification}

\subsection{Energy generation}\label{energy-generation}

\subsection{Animal feed}\label{animal-feed}

\chapter{Greenhouse gases}\label{greenhouse-gases}

\chapter{Animal welfare}\label{animal-welfare}

\chapter{Sustainable livestock
industry}\label{sustainable-livestock-industry}

\begin{quote}
``In essence, the conflict between livestock and the environment is a
conflict between different human needs and expectations.'' --- Henning
Steinfeld (FAO)
\end{quote}

\bibliography{book.bib,packages.bib}


\end{document}
